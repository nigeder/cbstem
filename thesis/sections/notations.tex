\documentclass[../main.tex]{subfiles}
\graphicspath{{figures/}{../figures/}}

\begin{document}
% \todo[inline,color=green!40]{完成符号说明
%   (filename: sections/notations)}


\todo[color=green!30]{引用符号可以直接\gls{e},点击会跳转到符号表,定义在section/notations.tex 中}

% 符号表
\printunsrtglossary[type=symbols,style=symbunitlong] % symbols

% 带单位的符号表
% \printunsrtglossary[type=symbols,style=symblong] % symbols without units



% % 也可以直接使用表格
% \begin{center}
%   \begin{tabularx}{\textwidth}{ c X }
%     \toprule[1.5pt]
%       \vspace{2pt}
%       \makebox[0.3\textwidth][c]{符号} &
%       \makebox[0.4\textwidth][c]{意义}    \\
%       \hline
%     \endhead
%
%     \bottomrule[1.5pt]
%     \vspace{2pt} % 加入一段垂直空白防止覆盖页码
%     \endfoot
%
%     % 表格正文
%     $N_i$ & 某一年第$i$月下雨天数 \\
%
%   \end{tabularx}
%
% \end{center}


\end{document}
